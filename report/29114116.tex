\documentclass[uplatex,12pt]{jsarticle}
\usepackage[dvipdfmx]{graphicx}
\usepackage{url}
\usepackage{listings,jlisting}
\usepackage{ascmac}
\usepackage{amsmath,amssymb}

%ここからソースコードの表示に関する設定
\lstset{
  basicstyle={\ttfamily},
  identifierstyle={\small},
  commentstyle={\smallitshape},
  keywordstyle={\small\bfseries},
  ndkeywordstyle={\small},
  stringstyle={\small\ttfamily},
  frame={tb},
  breaklines=true,
  columns=[l]{fullflexible},
  numbers=left,
  xrightmargin=0zw,
  xleftmargin=3zw,
  numberstyle={\scriptsize},
  stepnumber=1,
  numbersep=1zw,
  lineskip=-0.5ex
}
%ここまでソースコードの表示に関する設定

\title{知能プログラミング演習II 課題3}
\author{グループ8\\
  29114116 増田大輝\\
}
\date{2019年10月25日}

\begin{document}
\maketitle

\paragraph{提出物} rep3
\paragraph{グループ} グループ8

\paragraph{メンバー}
\begin{tabular}{|c|c|c|}
  \hline
  学生番号&氏名&貢献度比率\\
  \hline\hline
  29114003&青山周平&NoData\\
  \hline
  29114060&後藤拓也&NoData\\
  \hline
  29114116&増田大輝&NoData\\
  \hline
  29114142&湯浅範子&NoData\\
  \hline
  29119016&小中祐希&NoData\\
  \hline
\end{tabular}



\section{課題の説明}
\begin{description}
\item[必須課題4-1] まず,教科書3.2.1の「前向き推論」のプログラムと教科書3.2.2の「後向き推論」のプログラムとの動作確認をし,前向き推論と後ろ向き推論の違いを説明せよ.
また,実行例を示してルールが選択される過程を説明せよ.
説明の際には,LibreOfficeのDraw(コマンド soffice --draw)などのドロー系ツールを使ってp.106 図3.11やp.118 図3.12のような図として示すことが望ましい.
\item[必須課題4-2] CarShop.data , AnimalWorld.data 等のデータファイルを実際的な応用事例(自分達の興味分野で良い)に書き換えて,前向き推論,および後ろ向き推論に基づく質問応答システムを作成せよ.
どのような応用事例を扱うかは,メンバーで話し合って決めること.
なお,ユーザの質問は英語や日本語のような自然言語が望ましいが,難しければ変数を含むパターン等でも可とする.
\item[必須課題4-3] 上記4-2で実装した質問応答システムのGUIを作成せよ.
質問に答える際の推論過程を可視化できることが望ましい.
\item[発展課題4-4] 上記4-3で実装したGUIを発展させ,質問応答だけでなく,ルールの編集(追加,削除,変更)などについてもGUIで行えるようにせよ.
\end{description}

\section{課題4-1}
\begin{screen}
    まず,教科書3.2.1の「前向き推論」のプログラムと教科書3.2.2の「後向き推論」のプログラムとの動作確認をし,前向き推論と後ろ向き推論の違いを説明せよ.
    また,実行例を示してルールが選択される過程を説明せよ.
    説明の際には,LibreOfficeのDraw(コマンド soffice --draw)などのドロー系ツールを使ってp.106 図3.11やp.118 図3.12のような図として示すことが望ましい.
\end{screen}
本課題では,実装を行わないため,「実装」の節を省略することとする.
代わりに,前向き推論と後ろ向き推論について説明する節を設ける.
また,実行例を示した上でルールが選択される過程を説明するため,通常「実行例」の節を「実行例とルールが選択される過程の説明」として強調する.

\subsection{手法}
最初に,教科書3.2.1の「前向き推論」のプログラムと教科書3.2.2の「後向き推論」のプログラムの動作確認を行い,教科書を参考にそれぞれの推論の違いについて説明する.
次に,両者のプログラムの実行例を示しつつ,LibreOfficeのDrawによって作成した図を交えながらルールが選択される過程の説明を行う.

\newpage

\subsection{前向き推論と後ろ向き推論の違い}
前向き推論では,ルールインタプリタ内部の照合過程において,ルールの前件を満たすようなアサーションがワーキングメモリ内に存在する場合にルールが発火し,後件の評価やアクションの実行が行われる.
この一連の流れが,追加されるアサーションがなくなるまで繰り返し行われる.
前向き推論は,以下のステップにより行われる.
\begin{description}
    \item 1. ワーキングメモリ内のアサーションと選択されたルールの前件がマッチするかチェックする.
    \item 2. マッチしたルールの後件を具体化してアサーションとしてワーキングメモリに追加/アクションを実行する.
    \item 3. ルールが実行されなくなるまで1に戻って照合過程を繰り返し行う.
    \item 4. 新たに実行可能なルールが無くなった場合は終了する.
\end{description}
すなわち,ワーキングメモリ内部に蓄えられたアサーションは知識に当たり,これらの知識を用いて実行可能なルールから新たな知識が導出されることにより,逐次的な推論が行われる.
したがって,蓄えられた知識とルールから新たな知識を獲得し,その知識によって実行可能なルールがあれば,そこからさらなる知識を獲得することができるのである.
前向き推論では,与えられたアサーションとルールのパターンを照合することによって得られる全ての知識をワーキングメモリに追加する.
この特徴から,前向き推論では,網羅的に知識を獲得することができるというメリットがある.
その一方で,特定の知識のみを得たい際には,必要のない知識の照合も含めた推論を行う可能性が十分にあるため,非効率であるというデメリットがある. \\

後ろ向き推論では,与えられた仮説が現在のワーキングメモリ内のアサーション集合,すなわち保持している全ての知識において成立するか否かをルールを用いて調べることによって駆動する.
すなわち,以下の手順で推論が行われる.
\begin{description}
    \item 1. 与えられた仮説とマッチするルールの後件を探索する.
    \item 2. マッチしたルールの前件とワーキングメモリ内部のアサーションのマッチングを行う.
    \item 3. 成功した場合は仮説を真として判定. \\
             失敗した場合はマッチングに失敗した前件を新たな仮説とし,1から繰り返し,真偽判定を行う.
    \item 4. 全ての仮説が真と判断できなかった場合は最初の仮説を偽として判定.
\end{description}
以上をまとめると,後向き推論では得たい情報を仮説として,その仮説を導くためのルールを順に辿って行くことで,最終的に現在の知識から真偽や具体化を得る.
この特徴から,後向き推論では,特定の知識を導き出す際に,必要な知識のみを推論によって獲得するため効率的であるといえる.
その一方で,全ての知識についてルールが適用されるとは限らないので,知識獲得における網羅性はない. \\

以上が,前向き推論と後向き推論の特徴であり,それぞれの違いは推論の駆動のさせ方とその違いにより生じるメリット・デメリットが互いに相反している点にある.
具体的には,前向き推論ではワーキングメモリ内のアサーションの変化がルールが発火する原因となっており,後向き推論では仮説の変化がルールを発火させる原因となる.
以下にそれぞれの代表的な特徴を比較して違いがわかる表を示す.

\begin{table}[!hbt]
    \begin{tabular}{|l|l|l|}
    \hline
              & 照合過程でチェックする部分 & ルール発火の原因 \\ \hline
    前向き推論 & ルールの前件とアサーション & ワーキングメモリの変化 \\ \hline
    後向き推論 & ルールの後件と仮説     & 仮説の変化 \\ \hline
    \end{tabular}
    \caption{前向き推論と後向き推論の駆動方法の違い}
\end{table}

\begin{table}[!hbt]
    \begin{tabular}{|l|l|l|l|l|}
    \hline
              & メリット      & デメリット           \\ \hline
    前向き推論 & 網羅的に知識を獲得 & 余分な知識も獲得するため非効率 \\ \hline
    後向き推論 & 効率的な知識獲得  & 得られる知識が限定的      \\ \hline
    \end{tabular}
    \caption{前向き推論と後向き推論のメリット・デメリット}
\end{table}


\subsection{実行例とルールが選択される過程の説明}
\subsubsection{前向き推論の実行例について}
以下に前向き推論の実行結果を示す.
\begin{lstlisting}[caption=前向き推論の実行結果 , label=mid]
    ~/Programming2/Work4/ForwardChain
    ●java RuleBaseSystem                                                                                                                                                                                                      【 masuda-branch 】
    ADD:my-car is inexpensive
    ADD:my-car has a VTEC engine
    ADD:my-car is stylish
    ADD:my-car has several color models
    ADD:my-car has several seats
    ADD:my-car is a wagon
    CarRule1 [?x is inexpensive]->?x is made in Japan
    CarRule2 [?x is small]->?x is made in Japan
    CarRule3 [?x is expensive]->?x is a foreign car
    CarRule4 [?x is big, ?x needs a lot of gas]->?x is a foreign car
    CarRule5 [?x is made in Japan, ?x has Toyota's logo]->?x is a Toyota
    CarRule6 [?x is made in Japan, ?x is a popular car]->?x is a Toyota
    CarRule7 [?x is made in Japan, ?x has Honda's logo]->?x is a Honda
    CarRule8 [?x is made in Japan, ?x has a VTEC engine]->?x is a Honda
    CarRule9 [?x is a Toyota, ?x has several seats, ?x is a wagon]->?x is a Carolla Wagon
    CarRule10 [?x is a Toyota, ?x has several seats, ?x is a hybrid car]->?x is a Prius
    CarRule11 [?x is a Honda, ?x is stylish, ?x has several color models, ?x has several seats, ?x is a wagon]->?x is an Accord Wagon
    CarRule12 [?x is a Honda, ?x has an aluminium body, ?x has only 2 seats]->?x is a NSX
    CarRule13 [?x is a foreign car, ?x is a sports car, ?x is stylish, ?x has several color models, ?x has a big engine]->?x is a Lamborghini Countach
    CarRule14 [?x is a foreign car, ?x is a sports car, ?x is red, ?x has a big engine]->?x is a Ferrari F50
    CarRule15 [?x is a foreign car, ?x is a good face]->?x is a Jaguar XJ8
    apply rule:CarRule1
    Success: my-car is made in Japan
    ADD:my-car is made in Japan
    apply rule:CarRule2
    apply rule:CarRule3
    apply rule:CarRule4
    apply rule:CarRule5
    apply rule:CarRule6
    apply rule:CarRule7
    apply rule:CarRule8
    Success: my-car is a Honda
    ADD:my-car is a Honda
    apply rule:CarRule9
    apply rule:CarRule10
    apply rule:CarRule11
    Success: my-car is an Accord Wagon
    ADD:my-car is an Accord Wagon
    apply rule:CarRule12
    apply rule:CarRule13
    apply rule:CarRule14
    apply rule:CarRule15
    Working Memory[my-car is inexpensive, my-car has a VTEC engine, my-car is stylish, my-car has several color models, my-car has several seats, my-car is a wagon, my-car is made in Japan, my-car is a Honda, my-car is an Accord Wagon]
    apply rule:CarRule1
    apply rule:CarRule2
    apply rule:CarRule3
    apply rule:CarRule4
    apply rule:CarRule5
    apply rule:CarRule6
    apply rule:CarRule7
    apply rule:CarRule8
    apply rule:CarRule9
    apply rule:CarRule10
    apply rule:CarRule11
    apply rule:CarRule12
    apply rule:CarRule13
    apply rule:CarRule14
    apply rule:CarRule15
    Working Memory[my-car is inexpensive, my-car has a VTEC engine, my-car is stylish, my-car has several color models, my-car has several seats, my-car is a wagon, my-car is made in Japan, my-car is a Honda, my-car is an Accord Wagon]
    No rule produces a new assertion
\end{lstlisting}
実行結果の前半では,各推論システムの構成要素の初期化が行われている.
具体的には,3〜8行目においてワーキングメモリ内のアサーション集合が初期化され,9〜23行目においてはルールベースが初期化されている. \\
次に,24行目以降の推論の実行が行われている後半部分について説明する.
説明のポイントとして,ルールが実行されている部分に注目する. \\
まず,はじめにルールが実行されているのは,24〜26行目の部分である.
具体的には,ワーキングメモリ内のアサーション"my-car is inexpensive"によってCarRule1が実行された形になっている.
このルールの実行は以下の図の左側のようになる. \\
\begin{figure}[!hbt]
    \centering
    \includegraphics[scale=0.40]{images/forward_chaining_1.pdf}
    \caption{CarRule1の実行とワーキングメモリの変化}
\end{figure}
\newpage
また,上図の右側のワーキングメモリの変化からも読み取れるように,
ルールの実行結果として得られたアサーション"my-car is made in Japan"が実行後のワーキングメモリに追加されている.
このことは,ルールを実行する際の前件にマッチングできる可能性のある知識が増えたことを意味する. \\

続いて,33〜35行目に注目する.
ここでは,ワーキングメモリ内のアサーション"my-car is made in Japan"と"my-car has a VTEC engine"によってCarRule8が実行されている.
したがって,CarRule1の実行結果によって得られた知識を元に推論が駆動されている.
このルールの実行は以下の図の左側のようになる. \\
\begin{figure}[!hbt]
    \centering
    \includegraphics[scale=0.40]{images/forward_chaining_2.pdf}
    \caption{CarRule8の実行とワーキングメモリの変化}
\end{figure}
\newpage
ルールの実行結果として新たなアサーション"my-car is a Honda"がワーキングメモリに追加される. \\

同様に,38〜40行目に着目する.
ここでは,ワーキングメモリ内のアサーション5つを前件にマッチングさせてCarRule11を実行している.
このルールの実行は以下の図の左側のようになる.
\begin{figure}[!hbt]
    \centering
    \includegraphics[scale=0.40]{images/forward_chaining_3.pdf}
    \caption{CarRule11の実行とワーキングメモリの変化}
\end{figure}
\newpage
この推論の結果として,新たなアサーション"my-car is an Accord Wagon"が得られた. \\

その後,いかなるルールもマッチングせず,44行目で照合過程が一度終了する.
続く45行目では,現在のワーキングメモリ内部のアサーション集合の確認が行われている. \\
しかし,これで前向き推論のプログラムは終了しない.
46〜60行目にあるように,もう一度照合過程が行われている.
これは,前の照合過程において,あるルールを実行することによって得られた知識を用いて,
そのルールより以前にチェックされたルールが実行される可能性が残されているためである.
この操作は,前向き推論によって網羅的に知識が獲得されていくことを担保する.
ただし,今回の場合は,一周目で全ての知識が獲得されたために,二周目において新たに実行されるルールが存在せず,
62行目にあるように推論が終了している. \\
以上の前向き推論における全ルールの実行過程を示したものを以下に示す.
\begin{figure}[!hbt]
    \centering
    \includegraphics[scale=0.40]{images/forward_chaining_4.pdf}
    \caption{前向き推論の実行}
\end{figure}

\newpage


\subsubsection{後向き推論の実行例について}
以下に,後向き推論の実行結果を示す.
\begin{lstlisting}[caption=後向き推論の実行結果, label=mid]
    ~/Programming2/Work4/BackwordChain
    ●java RuleBaseSystem "?x is an Accord Wagon"                                                                                                                                                                              【 masuda-branch 】
    Hypothesis:[?x is an Accord Wagon]
    Success RULE
    Rule:CarRule11 [?x10 is a Honda, ?x10 is stylish, ?x10 has several color models, ?x10 has several seats, ?x10 is a wagon]->?x10 is an Accord Wagon <=> ?x is an Accord Wagon
    Success RULE
    Rule:CarRule7 [?x17 is made in Japan, ?x17 has Honda's logo]->?x17 is a Honda <=> ?x10 is a Honda
    Success RULE
    Rule:CarRule1 [?x18 is inexpensive]->?x18 is made in Japan <=> ?x17 is made in Japan
    Success WM
    his-car is inexpensive <=> ?x18 is inexpensive
    tmpPoint: 12
    Success:?x17 is made in Japan
    tmpPoint: -1
    Success RULE
    Rule:CarRule8 [?x37 is made in Japan, ?x37 has a VTEC engine]->?x37 is a Honda <=> ?x10 is a Honda
    Success RULE
    Rule:CarRule1 [?x38 is inexpensive]->?x38 is made in Japan <=> ?x37 is made in Japan
    Success WM
    his-car is inexpensive <=> ?x38 is inexpensive
    tmpPoint: 12
    Success:?x37 is made in Japan
    Success WM
    his-car has a VTEC engine <=> ?x37 has a VTEC engine
    tmpPoint: 19
    Success:?x10 is a Honda
    Success WM
    his-car is stylish <=> ?x10 is stylish
    tmpPoint: 3
    Success:?x10 is stylish
    Success WM
    his-car has several color models <=> ?x10 has several color models
    tmpPoint: 4
    Success:?x10 has several color models
    Success WM
    his-car has several seats <=> ?x10 has several seats
    tmpPoint: 5
    Success:?x10 has several seats
    Success WM
    his-car is a wagon <=> ?x10 is a wagon
    Yes
    {?x38=his-car, ?x37=his-car, ?x10=his-car, ?x=his-car}
    binding: {?x38=his-car, ?x37=his-car, ?x10=his-car, ?x=his-car}
    tmp: ?x, result:  his-car
    Query: ?x is an Accord Wagon
    Answer:his-car is an Accord Wagon
\end{lstlisting}


\subsection{考察}
セマンティックネットを構築する際に特に注目したのが,is-a関係による特性の継承である.
\verb|Chrono-Cross  =is-a=>  game| によって下位概念である「Chrono-Cross」から上位概念である「game」が関連付けられ,
\verb|game  =has-a=>  story| によって「game」に特性が紐づけられる.
結果として, \verb|Chrono-Cross  =has-a=>  story| が導出される実装から,知識表現とオブジェクト指向プログラミングの親和性の良さを体感した. \\
セマンティックネットを活用することによって,単純なデータの列挙から新たな知識を導出できることができる. \\
しかし,知識推論は一般から具体への方向にのみ働くことに注意が必要である.


% 参考文献
\begin{thebibliography}{99}
\bibitem{notty} Javaによる知能プログラミング入門 --著:新谷 虎松 \\
\bibitem{notty} LibreOffice Drawを用いた図版作成 --基礎プログラミング演習I資料(京都産業大学) (2019年11月12日アクセス)\\
\url{http://www.cse.kyoto-su.ac.jp/~oomoto/lecture/program/LibreOffice/draw/index-j.html} 
\bibitem{notty} LaTeXの表を生成できるサイト Tables Generator --muscle\_keisukeの日記 (2019年11月12日アクセス)\\
\url{http://muscle-keisuke.hatenablog.com/entry/2016/07/02/170205} 
\bibitem{notty} LaTeX Tables Generator --Tables Generator (2019年11月12日アクセス)\\
\url{http://www.tablesgenerator.com/latex_tables} \\
\end{thebibliography}

\end{document}